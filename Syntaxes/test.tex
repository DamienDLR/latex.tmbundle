\documentclass[letter]{article}


% Welcome to the Latex torture test file.  Please add your ugliest latex syntax examples 
% to the appropriate place in this file so we can make sure that Latex works consistently 
% for all of us.

% To make this document more colorful you may want to turn on other:Latex Theme

%preamble type stuff here


% 6/14/05  many of the commands below are functions with arguments.  When recursion is in place these should work nicely.

\foo[f]{bar}{$10 + 20$} Here is a simple single function with several parameters


\newcommand{\uint}{\mathop{\mathchoice%
{\rlap{\smash{$\displaystyle\intop\limits^{\uintbar}$}}}%
{\rlap{\smash{$\textstyle\intop\limits^{\uintbar}$}}}%
{\rlap{\smash{$\scriptstyle\intop\limits^{\uintbar}$}}}%
{\rlap{\smash{$\scriptscriptstyle\intop\limits^{\uintbar}$}}}%
}\!\int}

\smash{$foobar$}

This is a dollar sign \$ in the tex code, it should not start math mode.

This sentence contains some \verb!verbatim statements! in it.  It is also legal to have a sentence with \verb*|verbatim stuff|.  The * tells latex to make the spaces visible as squashed u characters like this \verb*~ ~.  Strangely the documentation says that \verb3this should work3 but  is getting matched by keyword.tex.general unless the three is followed by  a space (not too likely) In that case the \verb3 is runningaway but getting stopped by the end of line.

% Main document stuff here.
\begin{document}

\maketitle
\chapter{Chapter One}

\input{Chapter1.tex}

%itemize environment
\begin{itemize}
    \item Item one
    \item Item $2$
    \item item three with \textbf{bold text}
\end{itemize}

\section{Second one}
%enumerate environment
\begin{enumerate}
    \item Item one
    \item Item $2$
    \item item three with \textit{italic text}
\end{enumerate}

% the listing environment 
\begin{lstlisting}[caption=The Vertex Class,label=lst:vertex,float=htbp]
class Vertex:
    def __init__(self,num):
        self.id = num
        self.adj = []
        self.color = 'white'
        self.dist = sys.maxint
        self.pred = None
        self.disc = 0
        self.fin = 0
        self.cost = {}

    def addNeighbor(self,nbr,cost=0):
        self.adj.append(nbr)
        self.cost[nbr] = cost
\end{lstlisting}

\subsection{Subsection one}
This is a paragraph with some math $ \sin{3.1415} * \cos{3.1415} \angle  $.  This is an ordinary sentence after the math.  This works just fine with math mode including a rule for a bunch of math symbols.

\subsubsection{SubSubsection one point one}
%description environment
\begin{description}
    \item[Item one]  and some more.
    \item[Item $2$] and $2+2 = 4$
    \item[Item three] with \texttt{typewriter text}
\end{description}

\paragraph{Paragraph One}

\begin{inparaenum}
  \item increase impurity concentration or $L_{z}$ (impurity species);
  \item increase the recycling neutral concentration to increase $\Delta
    Q_{\text{at}}$ and $\Delta M_{\text{at}}$; 
  \item reduce the power flux
    ($P_{\text{sep}} / A_{\text{sep}})$ transported across the separatrix
    (e.g.  by increased radiation inside the separatrix, by reduced auxiliary
    heating or by increased plasma surface area); and \item increase the
    connection length $L_{\text{\abbr{SOL}}} = q_{95}\pi R$.
\end{inparaenum}
it can be shown from~(\ref{eq14-26}) that $\Delta M_{\text{at}} \sim$

Here is a fun example where a multi argument tex function can contain a latex environment!  to get the nested begin ends colored correctly we could include text.latex instead of source.tex in the meta.function.with-arg.tex rule.  Not sure thats a good idea though....
\begin{figure}[htbp]
  \centering
  \subfigure[]{
    \label{fig:dija}
    \begin{minipage}[b]{0.32\textwidth}
        \centering \includegraphics[width=5.2cm]{Graphs/dijkstraa}
      \end{minipage}}%%
  \subfigure[]{
    \label{fig:dijb}
    \begin{minipage}[b]{0.32\textwidth}
        \centering \includegraphics[width=5.2cm]{Graphs/dijkstrab}
      \end{minipage}}%%
  \label{fig:dijstep}
  \caption{Tracing Dijkstra's Algorithm}
\end{figure}


\begin{split}
  \Delta_n &=\frac{D_\bot }{\left( {{\Gamma_\bot } / {n_{\text{\abbr{SOL}}} }} \right)}\\
  \Delta_T &=\frac{\chi_\bot }{\left( {{Q_\bot } / {n_{\text{\abbr{SOL}}}
          T_{\text{\abbr{SOL}}} }} \right)-3{D_\bot } / {\Delta_n }}
\end{split}

% Make sure that the begin end patterns for section don't always match the first }
% Do section commands ever span more than one line????  I'm guessing not in 99% of the cases.
\section{$\mathcal{D}$-modules}

% part is tricky because it is used in the exam style and is also used in reports.
\part[4] $\sqrt[d]{n!}$  Hmmm, It would be nice to color math functions somehow....


\foobar some ordinary text

\begin{verbatim}
	This is some verbatim text.
\end{verbatim}

\end{document}